\section{Introduction}
Tokenization is the process of mapping raw text into a sequence of discrete units (tokens) that a model can embed and process. It influences vocabulary construction, sequence length, interpretability, and ultimately performance in downstream tasks \cite{liu_roberta_2019}. While subword tokenization has become a standard design choice for transformer-based models, its behavior is not neutral for morphologically rich languages.

Byte Pair Encoding (BPE) \cite{sennrich_neural_2016}, WordPiece \cite{schuster_japanese_2012}, and Unigram \cite{kudo_sentencepiece_2018} address out-of-vocabulary (OOV) words by representing rare forms as compositions of frequent subword units. This improves coverage and keeps vocabularies compact, but it can also split words in ways that cut across morpheme boundaries and blur grammatical function \cite{toraman_impact_2023, kaya_effect_2024}. Such fragmentation is especially relevant for agglutinative languages such as Turkish, where productive suffixation yields many surface forms from relatively few lemmas.

Turkish exhibits rich suffix morphology and systematic morphophonological alternations, including vowel harmony and consonant alternations at morpheme boundaries. For example, suffix allomorphs such as \textit{-lAr} (plural) and \textit{-dAn}/\textit{-tAn} (ablative) realize the same grammatical morpheme under different phonological contexts, and consonant alternations such as \textit{kitap} $\rightarrow$ \textit{kitabı} (p$\rightarrow$b before a vowel) create predictable surface variants of the same stem. Tokenizers that treat these variants as unrelated units can inflate redundancy and reduce the reuse of meaning-bearing units across inflections \cite{bayram_tokenization_2025}.

This paper introduces MFT, a linguistically informed hybrid tokenizer for Turkish. The method combines dictionary-driven morphological segmentation (roots and affixes), a normalization layer that maps common allomorphic variants to shared identifiers, and a controlled subword fallback for open-vocabulary coverage. Roots include a leading space to mark word boundaries, and an orthographic case token preserves capitalization without duplicating vocabulary entries.

We evaluate tokenization quality on TR-MMLU \cite{bayram_setting_2025} using two linguistic alignment metrics: Turkish Token Percentage (TR~\%) and Pure Token Percentage (Pure~\%), which quantify lexical/morphemic coverage and alignment with unambiguous root/affix boundaries, respectively \cite{bayram_tokenization_2025}. To address reviewer concerns about real-world applicability, we further include downstream evaluation on sentence embedding benchmarks. In a controlled random-initialization setting, we compare four matched models (MFT, CosmosGPT2, Mursit, and Tabi) on Semantic Textual Similarity (STS), MTEB-TR, and TurBLiMP, a Turkish benchmark of linguistic minimal pairs \cite{basar_turblimp_2025}.

Our contributions are threefold: we propose a morphology-first hybrid tokenizer for Turkish that is lossless via an explicit decoder; we provide a quantitative and qualitative evaluation of tokenization quality on TR-MMLU against widely used tokenizers; and we report controlled downstream comparisons across four matched embedding models to assess whether improved morpheme alignment translates into better sentence representations.
