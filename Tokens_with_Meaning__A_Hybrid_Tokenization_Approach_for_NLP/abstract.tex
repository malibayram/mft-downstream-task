\begin{abstract}

Tokenization shapes how language models perceive morphology and meaning in Natural Language Processing (NLP), yet widely used frequency-driven subword tokenizers (e.g., Byte Pair Encoding and WordPiece) can fragment morphologically rich and agglutinative languages in ways that obscure morpheme boundaries. We introduce a linguistically informed hybrid tokenizer for Turkish that combines (i) dictionary-driven morphological segmentation (roots and affixes), (ii) phonological normalization that maps allomorphic variants to shared identifiers, and (iii) a controlled subword fallback for out-of-vocabulary coverage. Concretely, our released Turkish vocabulary contains 20,000 root identifiers, 72 affix identifiers that cover 177 allomorphic surface forms, and 12,696 subword units; special tokens represent whitespace and orthographic case without inflating the vocabulary. We evaluate tokenization quality on TR-MMLU using two linguistic alignment metrics: Turkish Token Percentage (TR~\%), the proportion of produced tokens that correspond to Turkish lexical/morphemic units under our lexical resources, and Pure Token Percentage (Pure~\%), the proportion of tokens aligning with unambiguous root/affix boundaries. The proposed tokenizer reaches 90.29\% TR~\% and 85.80\% Pure~\% on TR-MMLU, substantially exceeding several general-purpose tokenizers. We further validate practical utility with downstream sentence embedding benchmarks under a strict \emph{random initialization} control to isolate tokenizer inductive bias. Across four matched models (MFT, CosmosGPT2, Mursit, and Tabi), MFT improves Semantic Textual Similarity Benchmark (STSb-TR) Pearson correlation from 33.58\% (Tabi) to 50.37\%, and achieves the strongest overall average on Massive Text Embedding Benchmark (MTEB-TR) and the strongest TurBLiMP linguistic sensitivity among the evaluated baselines.

\textbf{Keywords:} Tokenization, Morphologically Rich Languages, Morphological Segmentation, Byte Pair Encoding, Turkish NLP, Linguistic Integrity, Low-Resource Languages
\end{abstract}
