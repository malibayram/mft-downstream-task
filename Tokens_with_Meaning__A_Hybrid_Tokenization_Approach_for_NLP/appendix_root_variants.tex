% Auto-generated by scripts/generate_paper_appendix_tables.py
\subsection{Root Variant Sets}
\label{sec:appendix_root_variants}
Some root identifiers correspond to multiple surface strings (e.g., to unify common alternations). This appendix reports the distribution of root variant set sizes and provides representative examples.

\begin{table}[htbp]
\centering
\caption{Distribution of root variant set sizes in \texttt{kokler.json}.}
\label{tab:root_variant_dist}
\begin{tabular}{rr}
\toprule
Variant set size & \# Root IDs \\
\midrule
1 & 17779 \\
2 & 2211 \\
3 & 10 \\
\bottomrule
\end{tabular}
\end{table}

\small
\begin{longtable}{rp{12.2cm}}
\caption{Examples of root identifiers with multiple surface strings (top 150 by variant count).}\\
\label{tab:root_variant_examples}\\
\toprule
Root ID & Surface strings \\
\midrule
\endfirsthead
\toprule
Root ID & Surface strings \\
\midrule
\endhead
100 &  sıca,  sıcak,  sıcağ \\
101 &  soğu,  soğuk,  soğuğ \\
102 &  ufa,  ufak,  ufağ \\
103 &  çabu,  çabuk,  çabuğ \\
104 &  seyre,  seyrek,  seyreğ \\
105 &  katk,  katık,  katığ \\
106 &  seçk,  seçik,  seçiğ \\
107 &  tutk,  tutuk,  tutuğ \\
108 &  yükse,  yüksek,  yükseğ \\
109 &  küçü,  küçük,  küçüğ \\
110 &  bedr,  bedir \\
111 &  fırl,  fırıl \\
112 &  nezr,  nezir \\
113 &  nesh,  nesih \\
114 &  seğr,  seğir \\
115 &  kasr,  kasır \\
116 &  rükn,  rükün \\
117 &  vezn,  vezin \\
118 &  vehm,  vehim \\
119 &  hışm,  hışım \\
120 &  sath,  satıh \\
121 &  böğr,  böğür \\
122 &  fecr,  fecir \\
123 &  cürm,  cürüm \\
124 &  nesr,  nesir \\
125 &  zabt,  zabıt \\
126 &  nutk,  nutuk \\
127 &  fasl,  fasıl \\
128 &  hüseyn,  hüseyin \\
129 &  hazm,  hazım \\
130 &  nabz,  nabız \\
131 &  kayn,  kayın \\
132 &  nazm,  nazım \\
133 &  misl,  misil \\
134 &  zikr,  zikir \\
135 &  hicv,  hiciv \\
136 &  fıkh,  fıkıh \\
137 &  kesr,  kesir \\
138 &  defn,  defin \\
139 &  meyl,  meyil \\
140 &  umr,  umur \\
141 &  kibr,  kibir \\
142 &  hemfikr,  hemfikir \\
143 &  hüzn,  hüzün \\
144 &  cebr,  cebir \\
145 &  ehl,  ehil \\
146 &  vasf,  vasıf \\
147 &  ritm,  ritim \\
148 &  şükr,  şükür \\
149 &  uğr,  uğur \\
150 &  sihr,  sihir \\
151 &  nefs,  nefis \\
152 &  feth,  fetih \\
153 &  kutb,  kutup \\
154 &  asr,  asır \\
155 &  vakf,  vakıf \\
156 &  hacm,  hacim \\
157 &  benz,  beniz \\
158 &  küfr,  küfür \\
159 &  oğl,  oğul \\
160 &  özr,  özür \\
161 &  omz,  omuz \\
162 &  göğs,  göğüs \\
163 &  cism,  cisim \\
164 &  seyr,  seyir \\
165 &  burn,  burun \\
166 &  ilm,  ilim \\
167 &  sabr,  sabır \\
168 &  koyn,  koyun \\
169 &  ağz,  ağız \\
170 &  gönl,  gönül \\
171 &  nehr,  nehir \\
172 &  karn,  karın \\
173 &  ömr,  ömür \\
174 &  atf,  atıf \\
175 &  zihn,  zihin \\
176 &  şahs,  şahıs \\
177 &  devr,  devir \\
178 &  beyn,  beyin \\
179 &  emr,  emir \\
180 &  nesl,  nesil \\
181 &  şekl,  şekil \\
182 &  aln,  alın \\
183 &  resm,  resim \\
184 &  şehr,  şehir \\
185 &  izn,  izin \\
186 &  kısm,  kısım \\
187 &  kayd,  kayıt \\
188 &  fikr,  fikir \\
189 &  metn,  metin \\
190 &  asl,  asıl \\
191 &  hükm,  hüküm \\
192 &  keyf,  keyif \\
193 &  kayb,  kayıp \\
194 &  akl,  akıl \\
195 &  vakt,  vakit \\
196 &  keşf,  keşif \\
197 &  çağr,  çağır \\
198 &  ed,  et \\
199 &  çok,  çoğ \\
200 &  gerek,  gereğ \\
201 &  büyük,  büyüğ \\
202 &  sahib,  sahip \\
203 &  birlik,  birliğ \\
204 &  hayad,  hayat \\
205 &  özellik,  özelliğ \\
206 &  çocuk,  çocuğ \\
207 &  çeşid,  çeşit \\
208 &  açık,  açığ \\
209 &  takib,  takip \\
210 &  birçok,  birçoğ \\
211 &  gid,  git \\
212 &  cevab,  cevap \\
213 &  sağlık,  sağlığ \\
214 &  erkek,  erkeğ \\
215 &  değişik,  değişiğ \\
216 &  kaynak,  kaynağ \\
217 &  gerçek,  gerçeğ \\
218 &  seçenek,  seçeneğ \\
219 &  arac,  araç \\
220 &  belird,  belirt \\
221 &  destek,  desteğ \\
222 &  genc,  genç \\
223 &  yemek,  yemeğ \\
224 &  taleb,  talep \\
225 &  artık,  artığ \\
226 &  hesab,  hesap \\
227 &  kitab,  kitap \\
228 &  teknik,  tekniğ \\
229 &  içerik,  içeriğ \\
230 &  yaklaşık,  yaklaşığ \\
231 &  uzak,  uzağ \\
232 &  istek,  isteğ \\
233 &  sonuc,  sonuç \\
234 &  ortak,  ortağ \\
235 &  sürec,  süreç \\
236 &  grub,  grup \\
237 &  güc,  güç \\
238 &  ihtiyac,  ihtiyaç \\
239 &  düşük,  düşüğ \\
240 &  sebeb,  sebep \\
241 &  dörd,  dört \\
242 &  eksik,  eksiğ \\
243 &  müzik,  müziğ \\
244 &  bebek,  bebeğ \\
245 &  reng,  renk \\
246 &  kayded,  kaydet \\
247 &  gelecek,  geleceğ \\
248 &  günlük,  günlüğ \\
249 &  güvenlik,  güvenliğ \\
\bottomrule
\end{longtable}
\normalsize

