\section{Future Work}
\label{sec:future_work}

While we add downstream sentence embedding evaluation for Turkish (Section~\ref{sec:downstream}), several directions remain open.

\paragraph{Beyond Turkish.}
Our framework is language-agnostic in structure but requires language-specific resources (root and affix inventories, plus decoding rules). A critical next step is evaluating the same pipeline on additional morphologically rich languages (e.g., other Turkic languages, Finno-Ugric languages) to separate framework-level benefits from Turkish-specific choices.

\paragraph{Broader linguistic coverage.}
Turkish morphophonology includes alternations and exceptions beyond the subset covered by our current rules (e.g., additional consonant alternations such as k/\u{g}, gemination in loanwords, and harmony exceptions). Extending both the affix inventory and the decoder’s restoration rules---and reporting ablations---would clarify which linguistic components drive improvements.

\paragraph{Edge cases and losslessness.}
Capitalization beyond simple word-initial case (e.g., acronyms and mixed-case identifiers such as \texttt{HTTPServer}) remains imperfect under the current \texttt{<uppercase>} marker design. Improving case preservation without vocabulary inflation is an important practical refinement.

\paragraph{Efficiency characterization.}
Our TR-MMLU table reports processing time for the proposed tokenizer, but we do not provide comparable speed measurements for all baselines. Future work should include standardized throughput/latency evaluation across tokenizers and analyze the trade-off between linguistic purity, token count, and downstream latency.

\paragraph{Additional benchmarks.}
To strengthen the linguistic motivation, future experiments should evaluate models (not only tokenizers) on targeted linguistic benchmarks where morphology matters, alongside broad downstream suites (e.g., Turkish-specific or multilingual BLiMP-style evaluations).
